\documentclass[11pt,leqno]{article}
\usepackage[polish]{babel}
\usepackage[utf8]{inputenc}
%\usepackage{polski}
\usepackage[T1]{fontenc}
\frenchspacing
\usepackage{indentfirst}
% Pakiet do tabelek
\usepackage{multirow} 
\usepackage{graphicx}
\usepackage{pdflscape}

\begin{document}
%strona tytułowa
\begin{center}
\thispagestyle{empty}
{\Large Studencka Pracownia Inżynierii Oprogramowania}\\[0.5cm]
{\Large Zespół nr 8, IIUWr 2011/12}\\[2.5cm]

{Marcin Januszkiewicz, Piotr Sobczyk}\\[0.5cm]
{\huge Dokumentacja projektu \textbf{Tarantula}}\\[0.25cm]
{ Analiza rynku}\\[0.5cm]
\vfill
{\large Wrocław, \today}
\end{center}

\newpage
\tableofcontents
\newpage
\pagestyle{headings}

\section{Charakterystyka rynku}

\subsection{Ogólny opis}
Rynek programów wspierających tworzenie grafiki komputerowej jest bardzo szeroki i szybko rozwijający się. Moc kart graficznych wzrasta dynamicznie i coraz
bardziej zaawansowane obrazy, symulacje mogą zostać wygenerowane przy pomocy komputera. Podstawą rynku są producenci gier komputerowych, ale coraz częściej
animacje są wykorzystywane na stronach internetowych i programach uzytkowych.
Istnieje wiele komercyjnych lub częściowo komercyjnych programów wspomagających tworzenie grafiki komputerowej. Wiele z nich, ze względu na cenę, 
nie jest powszechnie używanych. Program ,,Tarantula'' ma na celu służyć właśnie tym programistom, a dochodowość przedsięwzięcia ma nie polegać na cenie
programu, a na powszechności jego użycia.

\subsection{Wielkość grupy docelowej}
W założeniu Tarantula ma być wsparciem programistów, co jasno definiuje wielkość rynku jaki może zdobyć. Co roku polskie uniwersytety opuszcza około
100 tysięcy osób z wykształceniem informatycznym (lub podobnym). To właśnie dla osób młodych stworzona została Tarantula.

\end{document}

