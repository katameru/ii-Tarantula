\documentclass[11pt,leqno]{article}
\usepackage[polish]{babel}
\usepackage[utf8]{inputenc}
%\usepackage{polski}
\usepackage[T1]{fontenc}
\frenchspacing
\usepackage{indentfirst}
% Pakiet do tabelek
\usepackage{multirow} 
\usepackage{graphicx}
\usepackage{pdflscape}
\renewcommand{\thesection}{\arabic{section}.}
\renewcommand{\thesubsection}{\arabic{section}.\arabic{subsection}.}
\renewcommand{\thesubsubsection}{\arabic{section}.\arabic{subsection}.\arabic{subsubsection}.}

\begin{document}
%strona tytułowa
\begin{center}
\thispagestyle{empty}
{\Large Studencka Pracownia Inżynierii Oprogramowania}\\[0.5cm]
{\Large Zespół nr 8, IIUWr 2011/12}\\[2.5cm]

{Marcin Januszkiewicz, Piotr Sobczyk}\\[0.5cm]
{\huge Dokumentacja projektu \textbf{Tarantula}}\\[0.25cm]
{ Analiza rynku}\\[0.5cm]
\vfill
{\large Wrocław, \today}
\end{center}

\newpage
\tableofcontents
\newpage


\section{Charakterystyka rynku}

\subsection{Ogólny opis}
\noindent
Rynek programów wspierających tworzenie grafiki komputerowej jest bardzo szeroki. Wikipedia podaje, że istnieje ponad sto niekomercyjnych bibliotek graficznych i niemal drugie tyle produktów komercyjnych. Podstawą rynku są twórcy gier komputerowych, ale animacje i wyświetlanie modeli są też wykorzystywane w programach użytkowych w celach wizualizacji. Ponieważ Tarantula została napisana z myślą o tworzeniu gier komputerowych, to programiści chcący napisać grę komputerową są najbardziej naturalną grupą odbiorców.

\subsection{Wielkość grupy docelowej}
\noindent
Najpopularniejsze forum internetowe przeznaczone temetyce tworzenia gier, {\it forums.indiegamer.com}, ma ponad 200 000 członków. Wliczając inne fora internetowe, wielkość grupy docelowej szacujemy na ponad 500 000 programistów. 
\subsection{Trendy rynku}
\noindent
W ostatnich latach znacząco wzrosło zainteresowanie grami wydawanymi przez małe, niezależne studia. Wraz z tym wzrostem zainteresowania i łatwiejszymi metodami dystrybucji, coraz więcej osób decyduje się na próbę napisania własnej gry. Choć rynek zdominowany jest przez kilka produktów (OGRE, Unreal Engine, Source Engine, id Tech, Cryengine), to programiści często szukają biblioteki która będzie najbardziej pasowała do ich potrzeb.



\end{document}

