	\documentclass[11pt,leqno]{article}
\usepackage[polish]{babel}
\usepackage[utf8]{inputenc}
%\usepackage{polski}
\usepackage[T1]{fontenc}
\frenchspacing
\usepackage{indentfirst}
% Pakiet do tabelek
\usepackage{multirow} 
\usepackage{graphicx}
\usepackage{pdflscape}
\renewcommand{\thesection}{\arabic{section}.}
\renewcommand{\thesubsection}{\arabic{section}.\arabic{subsection}.}
\renewcommand{\thesubsubsection}{\arabic{section}.\arabic{subsection}.\arabic{subsubsection}.}

\begin{document}
%strona tytułowa
\begin{center}
\thispagestyle{empty}
{\Large Studencka Pracownia Inżynierii Oprogramowania}\\[0.5cm]
{\Large Zespół nr 8, IIUWr 2011/12}\\[2.5cm]

{\large Marcin Januszkiewicz, Piotr Sobczyk}\\[0.5cm]
{\huge Dokumentacja projektu \textbf{Tarantula}}\\[0.25cm]
{\huge Słownik}\\[0.5cm]
\vfill
{\large Wrocław, \today}
\end{center}

\newpage


\section{Słownik}
\begin{description}
 \item[scena] --- abstrakcyjna reprezentacja wizualizowanych sytuacji. Scena może zawierać modele, na przykład drzewa i domy, oświetlające te modele {\bf źródła światła} i obserwujące wszystko {\bf kamery}.
 
 \item[model] --- komputerowa reprezentacja obiektu trójwymiarowego.


\item[trójkąt] --- podstawowa jednostka modelu trójwymiarowego, opisuje naprostszą rysowaną powierzchnię. Trójkąt opisywany jest przez trzy punkty w przestrzeni trójwymiarowej. Modele mogą składać się nawet z tysięcy trójkątów.

\item[tekstura] --- opis powierzchni modelu. Z reguły ogranicza się do jej wyglądu, ale może też zawierać informacje mające wpływ na liczenie oświetlenia.

\item[system cząsteczkowy] --- technika umożliwiająca symulację zdarzeń takich jak ogień, eksplozje, deszcz, co byłoby trudne do zrobienia korzystając z konwencjonalnych technik wizualizacji.

\item[mechanizm poziomu detali] --- element oprogramowania graficznego pozwalający na zmiejszenie wykorzystania zasobów modelu przez zmiejszenie liczby trójkątów, uproszczenie tekstury, niesymulowanie efektów lub nawet nierysowanie go w ogóle, jeżeli nie jest to potrzebne (np. obiekt położony jest bardzo daleko).

\item[model oświetlenia] --- zestaw reguł obliczania natężenia i koloru światła padającego na każdy rysowany obiekt.

\item[OpenGL] --- interfejs graficzny dostarczany przez firmę ARB, dostępny m.in. w systemach Windows, Linux, Mac OS.

\item[źródło światła] --- element sceny zawierający informacje potrzebne do obliczenia oświetlenia w scenie. Są to między innymi położenie, natężenie i kierunek padania światła, a także inne informacje zależne od wybranego modelu oświetlenia (na przykład, czy źródło światło jest punktem czy zachowouje się jak reflektor).

\item[kamera] --- element sceny zawierający informacje potrzebne do wyświetlenia sceny z 'punktu widzenia' tego elementu.
 \end{description}
\end{document}

