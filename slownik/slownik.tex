	\documentclass[11pt,leqno]{article}
\usepackage[polish]{babel}
\usepackage[utf8]{inputenc}
%\usepackage{polski}
\usepackage[T1]{fontenc}
\frenchspacing
\usepackage{indentfirst}
% Pakiet do tabelek
\usepackage{multirow} 
\usepackage{graphicx}
\usepackage{pdflscape}
\renewcommand{\thesection}{\arabic{section}.}
\renewcommand{\thesubsection}{\arabic{section}.\arabic{subsection}.}
\renewcommand{\thesubsubsection}{\arabic{section}.\arabic{subsection}.\arabic{subsubsection}.}

\begin{document}
%strona tytułowa
\begin{center}
\thispagestyle{empty}
{\Large Studencka Pracownia Inżynierii Oprogramowania}\\[0.5cm]
{\Large Zespół nr 8, IIUWr 2011/12}\\[2.5cm]

{\large Marcin Januszkiewicz, Piotr Sobczyk}\\[0.5cm]
{\huge Dokumentacja projektu \textbf{Tarantula}}\\[0.25cm]
{\huge Słownik}\\[0.5cm]
\vfill
{\large Wrocław, \today}
\end{center}

\newpage


\section{Słownik}
\begin{description}
 \item[scena] --- abstrakcyjna reprezentacja tego, co jest widoczne w wirtualnym świecie. Scena może zawierać modele na przykład drzewa, domy, oświetlające je źródła światła i obserwujące wszystko kamery.
 
 \item[model] --- komputerowa reprezentacja trójwymiarowego obiektu


\item[trójkąt] --- podstawowa jednostka modelu trójwymiarowego, opisuje naprostszą rysowalną powierzchnię. Trójkąt opisywany jest przez trzy punkty w przestrzeni trójwymiarowej. Modele mogą składać się nawet z tysięcy trójkątów.

\item[tekstura] --- opis powierzchni modelu, z reguły ogranicza się do jej wyglądu

\item[system cząsteczkowy] --- technika umożliwiająca symulację zdarzeń takich jak ogień, eksplozje, deszcz, co byłoby trudne do zrobienia korzystając z konwencjonalnych technik wizualizacji.

\item[silnik poziomu detali] --- element silnika graficznego pozwalający na zmiejszenie wykorzystania zasobów przez model przez zmiejszenie liczby jego trójkątów, uproszczenie tekstury, niesymulowaniu działających nań efektów lub nawet nierysowaniu go w ogóle, jeżeli nie jest to potrzebne (np. obiekt położony jest bardzo daleko).

\item[model oświetlenia] --- zestaw reguł pozwalający na obliczenie natężenia i koloru światła padającego na każdy rysowany obiekt.

\item[Direct3D] --- API graficzne dostarczane jako część biblioteki graficznej DirectX przez firmę Microsoft. Dostępne jedynie na platformie windows oraz w pewnym stopniu na konsolach Xbox.

\item[OpenGL] --- API graficzne dostarczane przez firmę ARB, dostępne m.in. na systemach Windows, Linux, Mac OS.
 \end{description}
\end{document}

