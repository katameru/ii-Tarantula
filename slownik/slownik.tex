	\documentclass[11pt,leqno]{article}
\usepackage[polish]{babel}
\usepackage[utf8]{inputenc}
%\usepackage{polski}
\usepackage[T1]{fontenc}
\frenchspacing
\usepackage{indentfirst}
% Pakiet do tabelek
\usepackage{multirow} 
\usepackage{graphicx}
\usepackage{pdflscape}

\title{\LARGE Dokumentacja projektu \textbf{Tarantula}\\
							Słownik}
\author{Marcin Januszkiewicz, Piotr Sobczyk}
\date{Wrocław, \today}

\begin{document}
\maketitle 
\newpage
\tableofcontents
\newpage
\pagestyle{headings}

\section{Słownik}

API - interfejs programowania aplikacji (ang. Application Pragramming Interface); zestaw reguł pozwalający dwóm programom komputerowym na komunikację.

optymalizacja - przekształcenie mające na celu redukcję wykorzystywanych zasobów, przy niewidocznej dla użytkownika różnicy (strasznie koślawe :D )

scena - abstrakcyjna reprezentacja tego, co jest widoczne w wirtualnym świecie. Scena może zawierać modele n.p. drzewa, domy, oświetlające je źródła światła i obserwujące wszystko kamery.

model - komputerowa reprezentacja trójwymiarowego obiektu

trójkąt - podstawowa jednostka modelu trójwymiarowego, opisuje naprostszą rysowalną powierzchnię. Trójkąt opisywany jest przez trzy punkty w przestrzeni trójwymiarowej. Modele mogą składać się nawet z tysięcy trójkątów.

tekstura - opis powierzchni modelu, z reguły ogranicza się do jej wyglądu

wykrywanie kolizji - proces pozwalający na określenie, czy dwa modele w przestrzeni się przecinają (kolidują) i dostarczający informacji o tej kolizji (czas kolizji i punkty przecięcia).

system cząsteczkowy - technika umożliwiająca symulację zdarzeń takich jak ogień, eksplozje, deszcz, co byłoby trudne do zrobienia korzystając z konwencjonalnych technik wizualizacji.

silnik poziomu detali - element silnika graficznego pozwalający na zmiejszenie wykorzystania zasobów przez model przez zmiejszenie liczby jego trójkątów, uproszczenie tekstury, niesymulowaniu działających nań efektów lub nawet nierysowaniu go w ogóle, jeżeli nie jest to potrzebne (np. obiekt położony jest bardzo daleko).

model oświetlenia - zestaw reguł pozwalający na obliczenie natężenia i koloru światła padającego na każdy rysowany obiekt.

Direct3D - API graficzne dostarczane jako część biblioteki graficznej DirectX przez firmę Microsoft. Dostępne jedynie na platformie windows oraz w pewnym stopniu na konsolach Xbox.

OpenGL - API graficzne dostarczane przez firmę ARB, dostępne m.in. na systemach Windows, Linux, Mac OS.

\end{document}

