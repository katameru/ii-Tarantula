\documentclass[11pt,leqno]{article}
\usepackage[polish]{babel}
\usepackage[utf8]{inputenc}
%\usepackage{polski}
\usepackage[T1]{fontenc}
\frenchspacing
\usepackage{indentfirst}
% Pakiet do tabelek
\usepackage{multirow} 
\usepackage{graphicx}
\usepackage{pdflscape}



\begin{document}

%strona tytułowa
\begin{center}
\thispagestyle{empty}
{\Large Studencka Pracownia Inżynierii Oprogramowania}\\[0.5cm]
{\Large Zespół nr 8, IIUWr 2011/12}\\[2.5cm]

{\large Marcin Januszkiewicz, Piotr Sobczyk}\\[0.5cm]
{\huge Dokumentacja projektu \textbf{Tarantula}}\\[0.25cm]
{\huge Analiza ryzyka}\\[0.5cm]
\vfill
{\large Wrocław, \today}
\end{center}

\newpage
\tableofcontents
\newpage

\section{Ryzyka i sposoby ich redukcji}
\noindent
Naturalnym czynnikiem wpływającym negatywnie na postęp realizacji projektu są wypadki losowe. Nie można całkowicie wyeliminować ryzyka związanego
z takimi zdarzeniami, jednak wiedząc uprzednio co zagraża powodzeniu projektu można zmniejszać szansę wystąpienia niekorzystnego zjawiska oraz minimalizować szkody
jakie takie zjawisko może wywołać. Poszczególne czynniki potencjalnie utrudniające realizację projektu zostały zamieszczone w tabeli. 
Dla każdego ryzyka zostało oszacowane prawdopobieństwo jego wystąpienia oraz wymieniono sposoby dzięki którym można zmniejszyć szkody. 
Wiedza, która posłużyła jako podstawę konstrukcji tej tabeli, pochodzi z doświadczeń szefów projektu, którzy brali wcześniej udział w projektach o~ podobnej skali.


\vspace{1 cm}
\hspace{-1.5cm}
 \begin{tabular}{|p{6cm}|p{1.3cm}|p{6cm}|}
  \hline
 Czynnik ryzyka & Szansa wystąpienia & Sposób minimalizacji strat \\ \hline
  Nie potrafiący analitycznie myśleć programista & 25\% & Krótki test na IQ przy rekrutacji \\ \hline
 Choroba członka projektu & 95\% & Zapas czasu na realizację projektu. \\ \hline
 Choroba więcej niż trzech członków zespołu jednocześnie & 30\% & Szczepienia ochronne na koszt firmy \\ \hline
 Nieznajomość wykorzystywanej technologii przez członków  projektu & 5\% & Kursy doszkalające. Zapas czasu na realizację projektu \\ \hline
Poważny błąd w fazie analizy & 3\% & Zewnętrzna konsultacja przed przejściem do fazy projektowania \\ \hline
Poważny błąd w fazie projektowania & 5\% & Zewnętrzna konsultacja przed rozpoczęciem implementacji \\ \hline
Niewielkie opóźnienie w realizacji którejkolwiek z faz projektu & 80\% & Zapas czasu na realizację każdej z faz projektu \\ \hline
Wysokie koszty zewnętrznych usług & 80\% & Rozpoznanie rynku. Rezerwa budżetowa \\ \hline
Niska wydajność pracowników & 80\% & Rezerwa budżetowa na premie motywacyjne. Stworzenie dobrej atmosfery pracy \\ \hline
Uszkodzenie dysków twardych, awaria sprzętu & 5\% & Regularne tworzenie kopii zapasowych, ubezpieczenie \\ \hline
 \end{tabular}



\end{document}

