\documentclass[11pt,leqno]{article}
\usepackage[polish]{babel}
\usepackage[utf8]{inputenc}
%\usepackage{polski}
\usepackage[T1]{fontenc}
\frenchspacing
\usepackage{indentfirst}
% Pakiet do tabelek
\usepackage{multirow} 
\usepackage{graphicx}
\usepackage{pdflscape}
\renewcommand{\thesection}{\arabic{section}.}
\renewcommand{\thesubsection}{\arabic{section}.\arabic{subsection}.}
\renewcommand{\thesubsubsection}{\arabic{section}.\arabic{subsection}.\arabic{subsubsection}.}

\begin{document}
 


%strona tytułowa

\begin{center}
\thispagestyle{empty}
{\Large Studencka Pracownia Inżynierii Oprogramowania}\\[0.5cm]
{\Large Zespół nr 8, IIUWr 2011/12}\\[2.5cm]

{Marcin Januszkiewicz, Piotr Sobczyk}\\[0.5cm]
{\huge Dokumentacja projektu \textbf{Tarantula}}\\[0.25cm]
{ Kalkulacja kosztów}\\[0.5cm]
\vfill
{\large Wrocław, \today}
\end{center}

\newpage
\tableofcontents
\newpage

\section{Opis jakościowy kosztów}

\section{Opis ilościowy kosztów}

\vspace{1 cm}
\hspace{-1.5cm}
{\small Tabela 1. Czynniki ryzyka z oceną szansy wystąpienia i sposobu minimalizacji strat.}
 \begin{tabular}{|p{10cm}|p{2cm}|}
  \hline
 Rodzaj kosztu & Szacunkowa wartość  \\ \hline
  2 programistów na 8 tygodni & 2*8*40*20 = 12,800 zł. \\ \hline
 Komputer do wykonania testów & 5000 zł. \\ \hline
 Karty graficzne do testów & 2000 zł. \\ \hline
 Kosz zewnętrznej ekspertyzy w trakcie projektu & 2*1500 = 3000 zł. \\ \hline
Rezerwa na awarie & 1000 zł. \\ \hline
Rezerwa na premie dla programistów & 3000 zł. \\ \hline
Rezerwa na nieprzewidziane koszty & 2500 zł. \\ \hline
Łączne koszty & 19,300 zł. \\ \hline
 \end{tabular}

\end{document}

