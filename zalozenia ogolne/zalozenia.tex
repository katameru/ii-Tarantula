\documentclass[11pt,leqno]{article}
\usepackage[polish]{babel}
\usepackage[utf8]{inputenc}
\usepackage{setspace}
\usepackage[T1]{fontenc}
\frenchspacing
\usepackage{indentfirst}


\begin{document}

%strona tytułowa
\begin{center}
\thispagestyle{empty}
{\Large Studencka Pracownia Inżynierii Oprogramowania}\\[0.5cm]
{\Large Zespół nr 8, IIUWr 2011/12}\\[2.5cm]

{\large Marcin Januszkiewicz, Piotr Sobczyk}\\[0.5cm]
{\huge Dokumentacja projektu \textbf{Tarantula}}\\[0.5cm]
{\huge Założenia ogólne}\\[0.5cm]
\vfill
{\large Wrocław, \today}
\end{center}
\newpage
\tableofcontents

\newpage


\newpage
\section{Wprowadzenie}

\subsection{Wstęp}
\noindent
Celem niniejszego dokumentu jest opis wymagań stawianych aplikacji ze względu na jej przeznaczenie i 
sposób użycia oraz określenie najważniejszych założeń realizacji projektu. 
Nie są tu poruszane kwestie implementacyjne, których szczegóły można znaleźć w osobnych dokumentach. // konkretne odsylacze

\subsection{Opis produktu}

\subsubsection{Czym jest Tarantula?}
\noindent
Tarantula jest biblioteką graficzną, to znaczy zbiorem funkcji i struktur (w rozumieniu programistycznym) mającym 
na celu ułatwienie i przyspieszenie korzystania przez programistów z możliwości współczesnych kart graficznych. Osiągane jest to przez abstrakcję od poziomu interfejsu kary graficznej i udostępnienie programistom takich rozwiązań jak modele, sceny, efekty graficzne, obliczenia oświetlenia i inne.

\subsubsection{Tło}
\noindent
Wraz ze wzrostem mocy i popularności nowoczesnych kart graficznych, zwiększają się możliwości tworzenia grafiki. 
Niestety metody programowania kart graficznych wymagają dużego nakładu pracy, ponieważ funkcje dostarczane przez producentów 
działają na bardzo niskim poziomie abstrakcji. Na obiekty rozumiane przez ten interfejs składają się jedynie trójwymiarowe modele zbudowane z trójkątów i płaskich tekstur. 
Wszystkie bardziej skomplikowane konstrukcje wymagają dużego wkładu własnego programisty. W związku z tym opracowano wiele programów 
ułatwiających korzystanie z tych możliwości kart graficznych. 

\subsubsection{Dlaczego stworzona została Tarantula?}
\noindent
Głównym celem stworzenia Tarantula jest wykorzystanie jej przez firmę przy produkcji własnych gier oraz narzędzi. 
Stworzenie własnej biblioteki ma w tym wypadku pewne korzyści nad wykorzystaniem jednego z wielu płatnych i bezpłatnych bibliotek 
dostępnych na rynku. W przypadku bibliotek komercyjnych przeszkodą jest oczywiście cena, natomiast w przypadku korzystania z darmowych 
bibliotek dochodzi konieczność zapewnienia zgodność z licencjami wszystkich komponentów użytych w bibliotece. Może to skutkować nawet 
brakiem możliwości sprzedaży gry lub oddaniu części dochodów twórcy biblioteki. W przypadku napisania własnej biblioteki istotna jest też możliwość dostosowanie szczegółowych 
możliwości i sposobu działania biblioteki do specyfiki produktów tworzonych przez firmę.

\section{Opis użytkownika}
\subsection{Dla kogo jest Tarantula?}
\noindent
Tarantula została napisana dla programistów tworzących aplikacje wyświetlające obrazy trójwymiarowe. Ponieważ Tarantula jest biblioteką a nie samodzielnym programem, 
nie jest bezpośrednio przydatna osobom, które nie zajmują się pisaniem programów.

\section{Szczegółowy opis produktu}

\subsection{Podstawowe możliwości}
\noindent
Do podstawowych możliwości oferowanych przez Tarantulę zaliczamy:
\begin{itemize}
 \item obsługę i ładowanie modeli i tekstur z popularnych formatów, dostarczanych na przykład przez programy 3d Studio Max i Blender. 
 Pozwala to na abstrakcję od poziomu karty graficznej, gdzie istnieją tylko wielokąty.
  \item wbudowane zarządzanie sceną
\item wykonanie elementarnych przekształceń graficznych
\item automatyczne optymalizacje takie jak usuwanie niewidocznych powierzchni, redukcja liczby trójkatów. Redukują one zasoby potrzebne na przechowywanie i generowanie grafiki.
\item wykrywanie kolizji
\item podstawowe wsparcie dla animacji
\item systemy cząsteczkowe oparte o realistyczne modele fizyczne
\item mechanizm poziomu detali umożliwiający tworzenie scen w dużej skali
\item zaawansowane modele oświetlenia (szeroki zakres dynamiczny (HDR), oświetlenie globalne (global illumination))
\item wykorzystanie możliwości najnowszych wersji API graficznych  - OpenGL 4.2
\end{itemize}


\subsection{Cechy produktu}
\noindent
Tarantula jest biblioteką napisaną w języku C++. Korzystanie z niej sprowadza się do dołączenia do pisanego zaimportowania odpowiednich plików nagłówkowych, 
które zawierają deklaracje klas i funkcji. Korzystając z dostarczonej funkcjonalności wystarczy przygotować okno do wyświetlania grafiki, 
uruchomić manadżera sceny i program jest gotowy do wyświetlania podstawowej grafiki.




\section{Podstawowe przypadki użycia}
Podstawowe przypadki użycia
\begin{itemize}
\item jako biblioteka w grze
\item jako element wyświetlający modele w przeglądarce zasobów graficznych
\item jako komponent programu do tworzenia podstawowych animacji
\end{itemize}


\end{document}
