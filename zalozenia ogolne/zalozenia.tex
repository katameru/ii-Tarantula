\documentclass[11pt,leqno]{article}
\usepackage[polish]{babel}
\usepackage[utf8]{inputenc}
\usepackage{setspace}
\usepackage[T1]{fontenc}
\frenchspacing
\usepackage{indentfirst}


\title{\LARGE Dokumentacja projektu \textbf{Tarantula}\\
							Założenia ogólne}
\author{Marcin Januszkiewicz, Piotr Sobczyk}
\date{Wrocław, \today}

\begin{document}

\maketitle 
\newpage
\pagestyle{headings}
\tableofcontents

\newpage


\newpage
\section{Wprowadzenie}

\subsection{Cel dokumentu założeń ogólnych}
Celem niniejszego dokumentu jest opis wymagań stawianych aplikacji, ze względu na jej przeznaczenie i 
sposób użycia oraz określenie najważniejszych założeń realizacji projektu. 
Nie są poruszane tu kwestie implementacyjne, których szczegóły można znaleźć w osobnych dokumentach.

\subsection{Ogólny opis produktu}
\subsubsection{Czym jest Tarantula?}
Tarantula jest silnikiem graficznym, to znaczy zbiorem funkcji i struktur (w rozumieniu programistycznym) mającym 
na celu ułatwienie i przyspieszenie korzystania przez programistów z możliwości współczesnych kart graficznych. 
Interfejs udostępniany przez karty graficznie jest bardzo niskiego poziomu - 
opiera się na manipulowaniu podstawowymi figurami geometrycznymi przy użyciu niezaawansowanych funkcji. 
Dlatego każdy programista wykorzystujący możliwości kart graficznych do nietrywialnych celów musi skorzystać z jakiegoś silnika graficznego.

\section{Opis użytkownika}
\subsection{Dla kogo jest stworzona Tarantula?}

\subsection{Dane statystyczne dotyczące użytkowników i rynku}

\section{Dalszy opis produktu}

\subsection{Podstawowe cechy}
Do podstawowych cech oferowanych przez Tarantulę zaliczamy:
\begin{itemize}
 \item obsługa i ładowanie modeli i tekstur z popularnych formatów, dostarczanych na przykład przez programy 3d Studio Max i Blender. 
 Pozwala to na abstrakcję od poziomu karty graficznej, gdzie istnieją tylko wielokąty.
  \item wbudowane zarządzanie sceną
\item wykonanie elementarnych przekształceń graficznych
\item automatyczne optymalizacje takie jak usuwanie niewidocznych powierzchni, redukcja liczby trójkatów. Redukują one zasoby potrzebne na przechowywanie i generowanie grafiki.
\item wykrywanie kolizji
\item podstawowe wsparcie dla animacji
\item systemy cząsteczkowe oparte o realistyczne modele fizyczne
\item silnik poziomu detali umożliwiający tworzenie scen w dużej skali
\item zaawansowane modele oświetlenia (szeroki zakres dynamiczny (HDR), oświetlenie globalne (global illumination))
\item wykorzystanie możliwości najnowszych wersji API graficznych  - OpenGL 4.2
\end{itemize}

\subsection{Określenie pozycji produktu na rynku}
Istnieje wiele programów wspomagających tworzenie grafiki komputerowej. ,,Tarantulę'' cechuje możliwość [coś tu musi być]

\subsection{Podsumowanie możliwości}


\section{Cechy produktu}

Tarantula jest biblioteką napisaną w języku C++. Korzystanie z niej sprowadza się do zaimportowania odpowiednich plików nagłówkowych, 
które zawierają deklaracje klas i funkcji. Korzystając z dostarczonej funkcjonalności wystarczy przygotować okno do wyświetlania grafiki, 
uruchomić manadżera sceny i program jest gotowy do wyświetlania podstawowej grafiki.


Głównym celem stworzenia silnika graficznego ,,Tarantula'' jest wykorzystanie go przez firmę przy produkcji własnych gier oraz narzędzi. 
Stworzenie własnego silnika ma w tym wypadku ma pewne korzyści nad wykorzystaniem jednego z wielu płatnych i bezpłatnych silników 
dostępnych na rynku. W przypadku silników komercyjnych przeszkodą jest oczywiście cena, natomiast w przypadku korzystania z darmowych 
silników dochodzi konieczność zapewnienia zgodność z licencjami wszystkich komponentów użytych w silniku. Może to skutkować nawet 
brakiem możliwości sprzedaży gry, lub oddaniu części dochodów twórcy silnika. Nie bez znaczenia jest też dostosowanie szczegółowych 
możliwości i sposobu działania silnika do specyfiki gier pisanych przez firmę.

\section{Podstawowe przypadki użycia}
Podstawowe przypadki użycia
\begin{itemize}
\item jako silnik w grze
\item jako element wyświetlający modele w przeglądarce zasobów graficznych
\item jako komponent programu do tworzenia podstawowych animacji
\end{itemize}

\section{Inne wymagania produktu}

\subsection{Spełniane normy}

\subsection{Wymagania stawiane systemowi}


\subsection{Licencjonowanie i instalacja}

\subsection{Wymagania efektywnościowe}

\end{document}
