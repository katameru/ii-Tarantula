\documentclass[11pt,leqno]{article}
\usepackage[polish]{babel}
\usepackage[utf8]{inputenc}
\usepackage{setspace}
\usepackage[T1]{fontenc}
\frenchspacing
\usepackage{indentfirst}


\begin{document}

%strona tytułowa
\begin{center}
\thispagestyle{empty}
{\Large Studencka Pracownia Inżynierii Oprogramowania}\\[0.5cm]
{\Large Zespół nr 8, IIUWr 2011/12}\\[2.5cm]

{Marcin Januszkiewicz, Piotr Sobczyk}\\[0.5cm]
{\huge Dokumentacja projektu \textbf{Tarantula}}\\[0.25cm]
{ Założenia ogólne}\\[0.5cm]
\vfill
{\large Wrocław, \today}
\end{center}
\newpage
\pagestyle{headings}
\tableofcontents

\newpage


\newpage
\section{Wprowadzenie}

\subsection{Wstęp}
Celem niniejszego dokumentu jest opis wymagań stawianych aplikacji, ze względu na jej przeznaczenie i 
sposób użycia oraz określenie najważniejszych założeń realizacji projektu. 
Nie są poruszane tu kwestie implementacyjne, których szczegóły można znaleźć w osobnych dokumentach.

\subsection{Opis produktu}

\subsubsection{Tło}

Wraz z rosnącymi mocą i popularnością nowoczesnych kart graficznych, zwiększają się możliwości tworzenia zaawansowanej grafiki. 
Niestety metody programowania kart graficznych wymagają dużego nakładu pracy, ponieważ funkcje dostarczane przez producentów 
działają na bardzo niskim poziomie abstrakcji. Obiekty rozumiane przez ten interfejs to jedynie trójwymiarowe modele z trójkątów i płaskie tekstury. 
Wszystkie bardziej skomplikowane konstrukcje wymagają dużego wkładu własnego progrmamisty. W związku z tym pojawiło się na rynku wiele programów 
ułatwiających korzystanie z tych możliwości kart graficznych. 


\subsubsection{Czym jest Tarantula?}
Tarantula jest silnikiem graficznym, to znaczy zbiorem funkcji i struktur (w rozumieniu programistycznym) mającym 
na celu ułatwienie i przyspieszenie korzystania przez programistów z możliwości współczesnych kart graficznych. 
Interfejs udostępniany przez karty graficznie jest bardzo niskiego poziomu - 
opiera się na manipulowaniu podstawowymi figurami geometrycznymi przy użyciu niezaawansowanych funkcji.
Dlatego każdy programista wykorzystujący możliwości kart graficznych do nietrywialnych celów musi skorzystać z jakiegoś silnika graficznego.

\section{Opis użytkownika}
\subsection{Dla kogo została stworzona Tarantula?}

Tarantula została stworzona dla osób tworzących aplikacje wyświetlające grafikę 3D. Ponieważ jest biblioteką a nie samodzielnym programem, 
nie jest bezpośrednio przydatna osobom które nie zajmują się pisaniem programów.

\section{Szczegółowy opis produktu}

\subsection{Podstawowe cechy}
Do podstawowych cech oferowanych przez Tarantulę zaliczamy:
\begin{itemize}
 \item obsługa i ładowanie modeli i tekstur z popularnych formatów, dostarczanych na przykład przez programy 3d Studio Max i Blender. 
 Pozwala to na abstrakcję od poziomu karty graficznej, gdzie istnieją tylko wielokąty.
  \item wbudowane zarządzanie sceną
\item wykonanie elementarnych przekształceń graficznych
\item automatyczne optymalizacje takie jak usuwanie niewidocznych powierzchni, redukcja liczby trójkatów. Redukują one zasoby potrzebne na przechowywanie i generowanie grafiki.
\item wykrywanie kolizji
\item podstawowe wsparcie dla animacji
\item systemy cząsteczkowe oparte o realistyczne modele fizyczne
\item silnik poziomu detali umożliwiający tworzenie scen w dużej skali
\item zaawansowane modele oświetlenia (szeroki zakres dynamiczny (HDR), oświetlenie globalne (global illumination))
\item wykorzystanie możliwości najnowszych wersji API graficznych  - OpenGL 4.2
\end{itemize}


\section{Cechy produktu}

Tarantula jest biblioteką napisaną w języku C++. Korzystanie z niej sprowadza się do zaimportowania odpowiednich plików nagłówkowych, 
które zawierają deklaracje klas i funkcji. Korzystając z dostarczonej funkcjonalności wystarczy przygotować okno do wyświetlania grafiki, 
uruchomić manadżera sceny i program jest gotowy do wyświetlania podstawowej grafiki.


Głównym celem stworzenia silnika graficznego ,,Tarantula'' jest wykorzystanie go przez firmę przy produkcji własnych gier oraz narzędzi. 
Stworzenie własnego silnika ma w tym wypadku ma pewne korzyści nad wykorzystaniem jednego z wielu płatnych i bezpłatnych silników 
dostępnych na rynku. W przypadku silników komercyjnych przeszkodą jest oczywiście cena, natomiast w przypadku korzystania z darmowych 
silników dochodzi konieczność zapewnienia zgodność z licencjami wszystkich komponentów użytych w silniku. Może to skutkować nawet 
brakiem możliwości sprzedaży gry, lub oddaniu części dochodów twórcy silnika. Nie bez znaczenia jest też dostosowanie szczegółowych 
możliwości i sposobu działania silnika do specyfiki gier pisanych przez firmę.

\section{Podstawowe przypadki użycia}
Podstawowe przypadki użycia
\begin{itemize}
\item jako silnik w grze
\item jako element wyświetlający modele w przeglądarce zasobów graficznych
\item jako komponent programu do tworzenia podstawowych animacji
\end{itemize}


\end{document}
