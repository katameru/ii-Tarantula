\documentclass[11pt,leqno]{article}
\usepackage[polish]{babel}
\usepackage[utf8]{inputenc}
\usepackage{setspace}
\usepackage[T1]{fontenc}
\frenchspacing
\usepackage{indentfirst}
\renewcommand{\thesection}{\arabic{section}.}
\renewcommand{\thesubsection}{\arabic{section}.\arabic{subsection}.}
\renewcommand{\thesubsubsection}{\arabic{section}.\arabic{subsection}.\arabic{subsubsection}.}

\begin{document}

%strona tytułowa
\begin{center}
\thispagestyle{empty}
{\Large Studencka Pracownia Inżynierii Oprogramowania}\\[0.5cm]
{\Large Zespół nr 8, IIUWr 2011/12}\\[2.5cm]

{\large Marcin Januszkiewicz, Piotr Sobczyk}\\[0.5cm]
{\huge Dokumentacja projektu \textbf{Tarantula}}\\[0.5cm]
{\huge Założenia ogólne}\\[0.5cm]
\vfill
{\large Wrocław, \today}
\end{center}
\newpage
\tableofcontents

\newpage


\newpage
\section{Wprowadzenie}

\subsection{Wstęp}
\noindent
Celem niniejszego dokumentu jest opis wymagań stawianych aplikacji ze względu na jej przeznaczenie i 
sposób użycia oraz określenie najważniejszych założeń realizacji projektu. 
Nie są tu poruszane kwestie implementacyjne, których szczegóły można znaleźć w osobnych dokumentach.

\subsection{Opis produktu}

\subsubsection{Czym jest Tarantula?}
\noindent
Tarantula jest biblioteką graficzną, to znaczy zbiorem funkcji i struktur (w~rozumieniu programistycznym) mającym 
na celu ułatwienie i przyspieszenie korzystania przez programistów z możliwości współczesnych kart graficznych. Osiągane jest to przez abstrachowanie od poziomu interfejsu karty graficznej i udostępnienie programistom takich rozwiązań, jak modele, sceny, efekty graficzne, obliczenia oświetlenia i inne.

\subsubsection{Tło}
\noindent
Wraz ze wzrostem mocy i popularności nowoczesnych kart graficznych, zwiększają się możliwości tworzenia grafiki. 
Niestety, metody programowania kart graficznych wymagają dużego nakładu pracy, ponieważ funkcje dostarczane przez producentów 
działają na bardzo niskim poziomie abstrakcji. Na obiekty rozumiane przez interfejs karty graficznej składają się jedynie trójwymiarowe modele zbudowane z trójkątów i płaskich tekstur. 
Wszystkie bardziej skomplikowane konstrukcje wymagają dużego wkładu własnego programisty. W~związku z tym opracowano wiele programów 
ułatwiających korzystanie z~zaawansowanych możliwości kart graficznych. 

\subsubsection{Dlaczego napisana została Tarantula?}
\noindent
Głównym powodem napisania Tarantula jest wykorzystanie jej przez firmę do produkcji własnych gier oraz narzędzi. 
Napisanie własnej biblioteki ma w tym wypadku pewne korzyści nad wykorzystaniem jednej z wielu płatnych i bezpłatnych bibliotek 
dostępnych na rynku. W przypadku bibliotek komercyjnych przeszkodą jest oczywiście cena, natomiast w przypadku korzystania z bibliotek niekomercyjnych dochodzi konieczność zapewnienia zgodności z licencjami wszystkich komponentów użytych w bibliotece. Może to skutkować nawet 
brakiem możliwości sprzedaży gry lub oddaniem części dochodów twórcy biblioteki. W przypadku napisania własnej biblioteki istotna jest też możliwość dostosowanie szczegółowych 
możliwości i sposobu działania biblioteki do specyfiki produktów tworzonych przez firmę.

\section{Opis użytkownika}
\subsection{Dla kogo jest Tarantula?}
\noindent
Tarantula została napisana dla programistów tworzących aplikacje wyświetlające obrazy trójwymiarowe. Ponieważ Tarantula jest biblioteką, a nie samodzielnym programem, 
nie jest bezpośrednio przydatna osobom, które nie zajmują się pisaniem programów.

\section{Szczegółowy opis produktu}

\subsection{Podstawowe możliwości}
\noindent
Do podstawowych możliwości Tarantuli zaliczamy:
\begin{itemize}
 \item obsługę i ładowanie modeli i tekstur z popularnych formatów, dostarczanych na przykład przez programy 3d Studio Max i Blender, co pozwala na odejście od poziomu abstrakcji karty graficznej, na którym istnieje tylko możliwość definiowania wielokątów;
  \item wbudowane zarządzanie sceną;
\item wykonywanie elementarnych przekształceń graficznych;
\item automatyczne optymalizacje zmniejszające zapotrzebowanie na pamięć i przetwarzanie, takie jak usuwanie niewidocznych powierzchni, redukcja liczby trójkatów;
\item {\bf systemy cząsteczkowe} napisane w oparciu o realistyczne modele fizyczne
\item {\bf mechanizm poziomu detali} umożliwiający tworzenie wielkoskalowych scen;
\item zaawansowane modele oświetlenia (szeroki zakres dynamiczny, oświetlenie globalne)
\item wykorzystanie możliwości najnowszych wersji interfejsu graficznego OpenGL
\end{itemize}


\section{Wykaz nazw podstawowych przypadków użycia}
Podstawowe przypadki użycia oprogramowania Tarantula:
\begin{itemize}
\item jako biblioteka w grze,
\item jako element wyświetlający modele w przeglądarce zasobów graficznych,
\item jako komponent programu do tworzenia statycznych obrazów.
\end{itemize}


\end{document}
