\documentclass[11pt,leqno]{article}
\usepackage[polish]{babel}
\usepackage[utf8]{inputenc}
%\usepackage{polski}
\usepackage[T1]{fontenc}
\frenchspacing
\usepackage{indentfirst}
% Pakiet do tabelek
\usepackage{multirow} 
\usepackage{graphicx}
\usepackage{pdflscape}


\begin{document}
%strona tytułowa

\begin{center}
\thispagestyle{empty}
{\Large Studencka Pracownia Inżynierii Oprogramowania}\\[0.5cm]
{\Large Zespół nr 8, IIUWr 2011/12}\\[2.5cm]

{Marcin Januszkiewicz, Piotr Sobczyk}\\[0.5cm]
{\huge Dokumentacja projektu \textbf{Tarantula}}\\[0.25cm]
{ Zaplecze sprzętowe i programowe}\\[0.5cm]
\vfill
{\large Wrocław, \today}
\end{center}

\newpage
\tableofcontents
\newpage


\section{Wymagania sprzętowe do użytkowania ,,Tarantuli''}

\begin{center}
 \begin{tabular}{|p{4cm}|p{3cm}|}
  \hline
 Część architektury komputerowej & Minimalne wymagania \\ \hline
  Procesor & 2 GH ??? \\ \hline
Architektura & 32 bity ??? \\ \hline
 Karta graficzna & ??? \\ \hline
Pamięć RAM & 2 GB \\ \hline
Dysk twardy & 30 MB ??? \\ \hline
 \end{tabular}
\end{center}

Dodatkowo, aby ,,Tarantula'' działała poprawnie potrzebny jest następujące oprogramowanie:
\begin{enumerate}
 \item Kompilator gcc w wersji 4.2 lub nowszej
  \item Biblioteki QT w wersji 4 (interfejs graficzny) ???
 \item Bazy danych MySQL ???
\end{enumerate}

\section{Narzędzia użyte w realizacji projektu. Informacja dla deweloperów}
Program ,,Tarantula'' składa się z dwóch części. Pierwsza, biblioteka wspomagająca tworzenie grafiki komputerowej,
została napisana w języku C. Druga, interfejs graficzny, została wykonana w technologii QT.

\end{document}

