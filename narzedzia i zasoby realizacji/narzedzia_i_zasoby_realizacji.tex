\documentclass[11pt,leqno]{article}
\usepackage[polish]{babel}
\usepackage[utf8]{inputenc}
%\usepackage{polski}
\usepackage[T1]{fontenc}
\frenchspacing
\usepackage{indentfirst}
% Pakiet do tabelek
\usepackage{multirow} 
\usepackage{graphicx}
\usepackage{pdflscape}
\renewcommand{\thesection}{\arabic{section}.}
\renewcommand{\thesubsection}{\arabic{section}.\arabic{subsection}.}
\renewcommand{\thesubsubsection}{\arabic{section}.\arabic{subsection}.\arabic{subsubsection}.}

\begin{document}
%strona tytułowa

\begin{center}
\thispagestyle{empty}
{\Large Studencka Pracownia Inżynierii Oprogramowania}\\[0.5cm]
{\Large Zespół nr 8, IIUWr 2011/12}\\[2.5cm]

{Marcin Januszkiewicz, Piotr Sobczyk}\\[0.5cm]
{\huge Dokumentacja projektu \textbf{Tarantula}}\\[0.25cm]
{\Large Zaplecze sprzętowe i programowe}\\[0.5cm]
\vfill
{\large Wrocław, \today}
\end{center}

\newpage
\tableofcontents
\newpage


\section{Wymagania sprzętowe do użytkowania ,,Tarantuli''}

\begin{center}
 \begin{tabular}{|p{6cm}|p{7cm}|}
  \hline
 Część architektury komputerowej & Minimalne wymagania \\ \hline
  Procesor & klasy Core 2 Duo lub lepszy  \\ \hline
Architektura & 32 bity \\ \hline
 Karta graficzna & NVIDIA GeForce seria 400 lub lepsze


 AMD Radeon seria 5000 lub lepsze
  \\ \hline
Pamięć RAM & 2 GB \\ \hline
Dysk twardy & 300 MB \\ \hline
 \end{tabular}
\end{center}

Ponadto, aby korzystać z Tarantuli potrzebny jest kompilator języka C++. Do pracy w systemie Linux zalecany jest kompilator gcc w wersji 4.2 lub lepszej, a do pracy w systemie Windows zalecany jest kompilator Visual C++ w wersji 9.0 lub lepszej.


\section{Narzędzia użyte w realizacji projektu. Informacja dla deweloperów}
Program ,,Tarantula'' składa się z dwóch części. Pierwsza, biblioteka wspomagająca tworzenie grafiki komputerowej,
została napisana w języku C. Druga, interfejs graficzny, została wykonana w technologii QT.

\end{document}

